% IMS
% Projekt
% Juraj Holub
% xholub40@stud.fit.vutbr.cz

\documentclass[a4paper, 11pt]{article}
\usepackage[utf8]{inputenc}
\usepackage[czech]{babel}
\usepackage[IL2]{fontenc}
\usepackage{times}
\usepackage[left=1.5cm,top=2.5cm,text={18cm,25cm}]{geometry}
\usepackage[unicode]{hyperref}
\usepackage{amsmath, amsthm, amsfonts, amssymb}
\usepackage{dsfont}
\setlength{\parindent}{1em}
\usepackage{hyperref}
\usepackage{graphicx}
\usepackage{float}
\usepackage{wrapfig}
\usepackage{listings}
\usepackage{url}
\usepackage{cite}

%\date{}

\lstset{
	basicstyle=\small\ttfamily,
}

\begin{document}
\begin{titlepage}
	\begin{center}
		\Huge
		\textsc{Fakulta informačních technologií \\
			Vysoké učení technické v~Brně} \\
		\vspace{\stretch{0.382}}
		{\LARGE
			IMS - Modelování a simulace \\ 
			\medskip 
			\Large{
				Ohrev užitkovej vody v rodinnom dome solárnym systémom.
			}
			\vspace{\stretch{0.618}}}
		\setlength{\parindent}{0.3em}\\
		{\Large 2019} \\
		{\Large Juraj Holub (xholub40)}\\
		{\Large Matej Parobek (xparob00)}
	\end{center}
\end{titlepage}

\tableofcontents
\newpage

\section{Úvod}
Stavebníctvo má v dnešnej dobe veľký dopad na životné prostredie. Spôsob získavania tepelnej energie pre ohrev obytných objektov pomocou alternatívnych zdrojov produkuje nezanedbateľne menšie množstvo CO$_2$ spalín. Táto práca analyzuje systém na ohrev užitkovej vody pre konkrétny rodinný dom. Simulačný model zhodnocuje dopad získavania tepelnej energie zo solárnych panelov na životné prostredie a návratnosť tejto investície v čase. 

\subsection{Zdroje infromácií a autori práce}
Konkrétna špecifikácia a hodnoty požiadaviek na tepelnú energiu v dome vychádzajú z nasledujúcej práce \cite{bc_solar_system}. Prácu vytvoril vedecký pracovník Energetického ústavu Fakulty strojního inženýrství VUT v Brne Ing. Ján Tuhovčák, Ph.D. ako záverečnú prácu a úspešne ju obhájil s klasifikáciou A. Cena na vybudovanie systému vychádza taktiež s tejto práce. Cena zemného plynu je získaná z aktuálneho cenníku fosilných palív pre November 2019 \cite{ceny_paliva}. Simulačný model vytvorili Juraj Holub A Matej Parobek na základe týchto informácií. 

\subsection{Validácia navrhovaného modelu}
Práca s ktorej model vychádza poskytuje ročné vyhodnotenie z hľadiska energetických nárokov objektu. Výsledky modelovej simulácie pre rovnaké časové obdobie sa zhodovali s týmito podkladmi. Z tohto hľadiska bol model vyhodnotený ako validný.


\section{Rozbor navrhovaného systému a použitých technológií}

V rodinnom dome je v prevádzke systém na ohrev vody pomocou zemného plynu s bežným kotlom. Aktuálna sadzba za plyn potrebný na vyprodukovanie 1 kWh je 1,71 CZK. Vyprodukovaná energia pomocou solárnych panelov je zadarmo ale počiatočná investícia na vybudovanie a nákup solárneho systému o ploche 6m$^2$ je 82 000 CZK. V rodinnom dome sa spotreba tepla na ohrev vody pohybuje okolo 3 713,079 kWh za rok. Spotreba je rovnomerná po celý rok, nezávislo od ročného obdobia. Naopak produkcia tepelnej energie pomocou solárnych panelov je závislá na ročnom období. Pre daný solárny panel a rodinný dom miestom v Brne je vzhľadom na počet dní v mesiaci, strednú teplotu vzduchu, pomernú dĺžku slnečného svitu, sklon kolektorov (30$^{\circ}$) a ich orientáciu (juh), produkcia tepla pre jednotlivé mesiace zobrazená v tabuľke \ref{tab:mesacna_produkcia_solar}.

\begin{table}[h]
	\centering
	\begin{tabular}{|c|c|c|}
		\hline
		\textbf{Mesiac} & \textbf{\begin{tabular}[c]{@{}c@{}}Vyprodukované teplo\\ {[}kWh{]}\end{tabular}} & \textbf{\begin{tabular}[c]{@{}l@{}}Počet dní\\ v mesiaci\end{tabular}} \\ \hline
		I. & 37,232 & 31 \\ \hline
		II. & 112,367 & 28 \\ \hline
		III. & 269,929 & 31 \\ \hline
		IV. & 370,764 & 30 \\ \hline
		V. & 617,854 & 31 \\ \hline
		VI. & 717,653 & 30 \\ \hline
		VII. & 768,901 & 31 \\ \hline
		VIII. & 601,239 & 31 \\ \hline
		IX. & 418,186 & 30 \\ \hline
		X. & 191,496 & 31 \\ \hline
		XI. & 58,362 & 30 \\ \hline
		XII. & 16,692 & 31 \\ \hline
	\end{tabular}
	\caption{Vyprodukované teplo pomocou solárnych panelov pre jednotlivé mesiace.}
	\label{tab:mesacna_produkcia_solar}
\end{table}

Ak solárna energia v danom mesiaci nepostačuje na pokrytie spotrebovaného tepla tak sa požadovaná energia získava sekundárnym zdrojom, ktorým je plynový kotol. Na druhej strane, prebitky solárnej energie sa v navrhovanom systéme vôbec nevyužívajú. Navrhovaný solárny systém neprodukuje žiadne spaliny CO$_2$. Naproti tomu, spaľovanie plynu produkuje 202g CO$_2$ spalín \footnote{Zdroj \href{https://www.oplyne.info/ecology/porovnanie-produkcie-znecistujucich-latok-so2-tzl-nox-co-a-sklenikoveho-plynu-co2-vyprodukovanych-spalinami-v-rodinnom-dome-vykurovanie-drevom-ciernym-hnedym-uhlim-a-zemnym-plynom/}{https://www.oplyne.info/ecology/porovnanie-produkcie-znecistujucich-latok...}}  na 1 kWh vyprodukovaného tepla. Podľa referenčnej práce sú emisie spojené s vybudovaním solárneho systému porovnateľné s emisiami na vybudovanie pôvodného systému. S tohto dôvodu emisie spojené s vybudovaním systému táto práca neuvažuje. 

\subsection{Postup požitý pre vytvorenie modelu}

Zo získaných vstupných informácií bol vytvorneý abstraktný model (IMS\cite{ims_slides} slide 9.) vo forme Petriho siete (IMS\cite{ims_slides} slide 123.). K nemu bol vytvorený ekvivalentný simulačný model (IMS\cite{ims_slides} slide 44.) v programovacom jazyku C++ za použitia knihovny \textbf{SIMLIB}\footnote{Project SIMLIB: \href{http://www.fit.vutbr.cz/~peringer/SIMLIB/.cs}{http://www.fit.vutbr.cz/~peringer/SIMLIB/.cs}} . Knihovna bola zvolená s ohľadom na zložitoť modelu. Použitie robustnejšej knihovny by vzhľadom na náročnosť abstraktného modelu bolo neprimerané. Táto knihovna poskytuje základné prostriedky pre diskrétne modelovanie (IMS\cite{ims_slides} slide 44.) ako sú procesy (IMS\cite{ims_slides} slide 121.) alebo obslužné linky (IMS\cite{ims_slides} slide 138.) a to pomocou prostriedkov Objektovo Orientovaného Programovania (OOP).

\subsection{Povod použitých technológií}
Na vytvorenie Petriho sieťe boli využité postupy preberané na predmete IMS\cite{ims_slides} v kapitole \textit{Diskrétní simulace}. Simulačný model bol implementovaný v jazyku C++ za použitia OOP abstrakcie a funkcionality zo štandardnej knihovny pre štandard z roku 2014. Program je prekladaný pomocou GNU C++ prekladača \texttt{g++} \footnote{GNU project \href{https://gcc.gnu.org/}{https://gcc.gnu.org/}} . Knihovnu SIMLIB  je využívaná pod licenciou GNU LGLP \footnote{GNU Lesser General Public License \href{https://www.gnu.org/licenses/lgpl-3.0.html}{https://www.gnu.org/licenses/lgpl-3.0.html}} . 

\section{Konceptuálny model}
Na základe rozboru navrhovaného systému bol vytvorený konceptuálny model (IMS\cite{ims_slides} slide 48.) popísaný v tejto kapitole.  Najmenšia časová jednotka v modeli je jeden deň. Takáto jednotka bola zvolená preto, že vyprodukované solárne zisky v systéme máme dostupné vždy pre časové obdobie jeden mesiac. Pre zmysluplné zhodnotenie výstupov musí model simulovať čas minimálne v rádoch rokov. Je tomu tak preto lebo na základe rozboru\footnote{Zdroj\cite{bc_solar_system} viď. strana 41} je návratnosť systému približne v období 35 rokov a menšie časové obdobie ako roky by teda neprinieslo hodnotné informácie. 

Vyprodukovaná tepelná energia je reprezentovaná procesmi, kde 1 kWh tepla vyprodukovaná alebo spotrebovaná za deň je reprezentovaná jedným a viac procesmi (počet procesov definuje uživateľ). Väčšie množstvo procesov reprezentujúcich jednu jednotku tepelnej energie zvyšuje presnosť simulácie ale taktiež zvyšuje výpočetnú náročnosť. 


\subsection{Petriho sieť} \label{petri_net_section}
Konceptuálny model reprezentovaný pomocou Petriho siete je priložený na obrázku \ref{obr_petri_net}. Stav (IMS\cite{ims_slides} slide 123.) \textit{Nový mesiac} má na začiatku jeden čakajúci proces. Tento proces okamžite prechádza do stavu \textit{Začiatok mesiaca}, kde je pozdržaný presne 1 mesiac a následne sa vracia do stavu \textit{Nový mesiac}. Prechod z \textit{nového mesiaca} do \textit{začiatok mesiaca} indukuje moment kedy v simulačnom čase začína nový kalendárny mesiac. Na začiatku nového mesiaca sa vygeneruje \textit{m} procesov do stavu \textit{Zdroj solárnej energie}, kde \textit{m} je rovné množstvu kWh tepla vyprodukovaného v daný mesiac solárnym panelom. Táto energia sa rozloží na celý mesiac pomocou časovaného prechodu s rovnomerným rozložením (IMS\cite{ims_slides} slide 89.) na intervale pokrývajúcom celý mesiac v jednotkách dňov. Následne každý proces reprezentujúci solárnu energiu prioritne prejde do stavu \textit{Spotrebovaná solárna energia} a to práve vtedy ak ešte nebola pokrytá denná spotreba energie. Množstvo denne požadovanej energie určuje linka \textit{Denná spotreba} s kapacitou \textit{n}, kde \textit{n} reprezentuje dennú spotrebu energie na ohrev vody. Procesy ktoré linku obsadia ju po 1 dni vždy uvoľnia. Ak je linka v daný deň už plne obsadená, tak proces prechádza do stavu \textit{Prebytočná solárna energia}. Tento stav reprezentuje solárnu energiu, ktorá nemá byť ako využitá. Naopak ak v danom dni nebolo vyprodukované dostatočné množstvo solárnej energie na pokrytie dennej spotreby, tak na konci dňa na linke \textit{Denná spotreba} ostáva nevyužitá voľná kapacita. Táto kapacita na konci dňa reprezentuje množstvo energie, ktoré nebolo dodané solárnym systémom a preto táto energia musela byť získaná spaľovaním zemného plynu.


\begin{figure}[h] 
	\centering
	\includegraphics[width=.8\paperwidth]{petri_net.pdf}
	\caption{Navrhnutý konceptuálny model vo forme Petriho siete.}
	\label{obr1}
\end{figure} \label{obr_petri_net}

\section{Architektúra programu}
Priebeh simulácie je veľmi závislý od simulačného času a to špecificky od mesiaca, ktorý je aktuálne simulovaný. Architektúra programu preto implementuje špeciálnu datovú štruktúru ktorá uchováva informácie o aktuálnom mesiaci v simulačnom čase. Tok jednej jednotky solárnej energie v priebehu mesiaca je simulovaný \textit{n} procesmi, kde \textit{n >= 1} (\textit{n} určuje uživateľ). Každý takýto proces vznikne a zanikne v rámci jedného mesiaca. Množstvo aktuálne vygenerovaných procesov opäť závisí od aktuálneho mesiaca. Každý proces (na základe miesta v petriho sieti) pridáva v priebehu simulácie datá do datovej štruktúri, ktorá zaznamenáva štatistické informácie o priebehu simulácie. Po uplinutí simulačného času po ktorý mala simulácia bežať je zo získaných štatistík dopočítané potrebné množstvo energie vyprodukovanej spaľovaním zemného plynu, ceny za vykurovanie a vyprodukované emisie.

\subsection{Ročný cyklus}
Ako popisuje sekcia \ref{petri_net_section}, miesto \textit{Nový mesiac} je obsadené 1 procesom vždy na začiatku nového kalendárneho mesiaca. Toto chovanie zabezpečuje proces, ktorý sa vždy uspí na mesiac. Avšak jednotlivé mesiace v roku sa líšia počtom dní. Preto je v rámci celého programu dostupná datová štruktúra, ktorá uchováva aktuálne prebiehajúci mesiac roku, pričom na začiatku je iniciovaná prvým mesiacom každého kalendárneho roku. Pri vstupe do miesta \textit{Nový mesiac}, tento proces vždy nastavený ďalší kalendárny mesiac v roku. Toto chovanie sa cyklicky opakuje po uplinutí roku. Táto štruktúra obsahuje pre každý mesiac príslušný počet dní a taktiež množstvo vyprodukovanej solárnej energie v danom mesiaci. Proces mesiaca sa teda vždy uspí na počet dní príslušný aktuálnemu mesiacu a zároveň vygeneruje príslušné množstvo procesov solárnej energie.

\subsection{Solárna energia}\label{solar_energy_proces}
Proces reprezentujúci solárnu energiu sa po svojom vzniku uspí a to na dobu vygenerovanú generátorom pseudonáhodných čísel (IMS\cite{ims_slides} slide 167.) s rovnomerným rozložením \textit{R(0, b-1)}, kde \textit{b} je počet dní v aktuálnom mesiaci. Takto sa spotreba energie rovnomerne rozloží na celý mesiac. Keď je proces aktívny, pokúsi sa obsadiť jedno miesto v linke \textit{Denná spotreba}. Linka má kapacitu rovnú celkovej dennej spotrebe energie na ohrev vody. Ak je linka dostupná, tak ju proces obsadí, uspí sa na 1 deň a následne uvoľní linku a zanikne.

\subsection{Používanie programu}
Simuláčný čas programu vždy začína v čase 0 a jeho dĺžku môže uživateľ nastaviť v rokoch. Ďalej môže uživateľ nastaviť množstvo spotrebovanej solárnej energie za rok a to v jednotkách kWh. Taktiež môže uživateľ stanoviť množstvo procesov reprezentujúcich 1 kWh tepelnej energie. Uživateľom sa doporučuje voľiť množstvo procesov na základe mesačnej spotreby. Mesačné pohyby energie v systéme sú v rádoch stoviek kWh. Cena a emisie pre menšie jednotky tepelnej energie sú zanedbateľne malé. Napríklad pre mesačne vyprodukovaných 500.4 kWh (s voľbou 1 proces na 1 kWh tepla) vznikne 500 procesov. Ak by sme chceli reprezentovať energiu o ešte o jeden rád presnejšie (10 procesov na 1 kWh) tak by pre 500.4 kWh vzniklo 5004 procesov. Výpočetná náročnosť by teda vzrástla 10-násobne ale výpočet by bol presnejší len o 0.4 kWh čo znamená, že výpočet sa spresní približne o 2\%.

Program sa dá preložiť spustením priloženého \texttt{Makefile} a následne spustiť:\\
\texttt{./ims-project -y 30 -e 3700 -p 6} \\
Argument \texttt{-y} definuje počet odsimulovaných rokov,argument \texttt{-e} definuje ročnú spotrebu energie a \texttt{-p} množstvo procesov na 1 kWh energie.

\section{Simulačné experimenty}
V tejto sekcii sa zhodnotí dopad produkcie tepelnej energie zo solárnych panel na životné prostredie v porovnaní s energiou získanou spaľovaním zemného plynu. Na vytvorený model sa aplikovalo niekoľko experimentov s cieľom zistiť koľko ročnú investíciu predstavuje využitie tohto systému.

\subsection{Experiment 1}
pustíme to na 1 rok a overíme validitu  modelu

\subsection{Experiment 2}
pustíme to na rozne vela rokov a budeme sledovat navratnost v case

\subsection{Experiment 3}
zvýšime množstvo vyproduk energ a sledujeme v čase

\subsection{Experiment 4}
znížime množstvo vyproduk energ a sledujeme v čase

\section{Záver}

\newpage
\bibliographystyle{czechiso}
\bibliography{bib}

\end{document}
